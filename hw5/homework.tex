\documentclass[12pt]{article}

\usepackage[margin=1in]{geometry}
\usepackage{amsmath,amsthm,amssymb}
\usepackage{tikz} % for drawing stuff
\usepackage{xcolor} % for \textcolor{}
\usepackage{readarray} % for \getargsC{}
\usepackage[utf8]{inputenc}
\usepackage[T1]{fontenc}
\usepackage{hyperref}

% Colors for players
\definecolor{darkred}{rgb}{0.64,0,0}
\definecolor{darkcyan}{rgb}{0,0.55,0.55}
\newcommand{\rowcolor}[1]{\textcolor{darkred}{#1}}
\newcommand{\columncolor}[1]{\textcolor{darkcyan}{#1}}
% Normal-form game
% \nfgame{T B L R TLR TLC BLR BLR TRR TRC BRR BRC}
\newcommand{\nfgame}[1]{%
\getargsC{#1}
\begin{tikzpicture}[scale=0.65]
\node (RT) at (-2,1) [label=left:\rowcolor{\argi}] {};
\node (RB) at (-2,-1) [label=left:\rowcolor{\argii}] {};
\node (CL) at (-1,2) [label=above:\columncolor{\argiii}] {};
\node (CR) at (1,2) [label=above:\columncolor{\argiv}] {};
\node (RTL) at (-1.4,0.6) {\rowcolor{\argv}}; % top/left row player payoff
\node (CTL) at (-0.6,1.4) {\columncolor{\argvi}}; % top/left column player payoff
\node (RBL) at (-1.4,-1.4) {\rowcolor{\argvii}};
\node (CBL) at (-0.6,-0.6) {\columncolor{\argviii}};
\node (RTR) at (0.6,0.6) {\rowcolor{\argix}};
\node (CTR) at (1.4,1.4) {\columncolor{\argx}};
\node (RBR) at (0.6,-1.4) {\rowcolor{\argxi}};
\node (CBR) at (1.4,-0.6) {\columncolor{\argxii}};
\draw[-,very thick] (-2,-2) to (2,-2);
\draw[-,very thick] (-2,0) to (2,0);
\draw[-,very thick] (-2,2) to (2,2);
\draw[-,very thick] (-2,-2) to (-2,2);
\draw[-,very thick] (0,-2) to (0,2);
\draw[-,very thick] (2,-2) to (2,2);
\draw[-,very thin] (-2,2) to (0,0);
\draw[-,very thin] (0,0) to (2,-2);
\draw[-,very thin] (-2,0) to (0,-2);
\draw[-,very thin] (0,2) to (2,0);
\end{tikzpicture}}
% \nfgame{T B L R TLR TLC BLR BLR TRR TRC BRR BRC}
\newcommand{\nfgamebig}[1]{%
\getargsC{#1}
\begin{tikzpicture}[scale=0.65]
\node (RT) at (-4,1) [label=left:\rowcolor{\argi}] {};
\node (RB) at (-4,-1) [label=left:\rowcolor{\argii}] {};
\node (CL) at (-1,4) [label=above:\columncolor{\argiii}] {};
\node (CR) at (1,4) [label=above:\columncolor{\argiv}] {};
\node (RTL) at (-2.4,0.6) {\rowcolor{\argv}}; % top/left row player payoff
\node (CTL) at (-1.6,3.4) {\columncolor{\argvi}}; % top/left column player payoff
\node (RBL) at (-2.4,-3.4) {\rowcolor{\argvii}};
\node (CBL) at (-1.6,-0.6) {\columncolor{\argviii}};
\node (RTR) at (1.6,0.6) {\rowcolor{\argix}};
\node (CTR) at (2.3,3.4) {\columncolor{\argx}};
\node (RBR) at (1.6,-3.4) {\rowcolor{\argxi}};
\node (CBR) at (2.4,-0.6) {\columncolor{\argxii}};
\draw[-,very thick] (-4,-4) to (4,-4);
\draw[-,very thick] (-4,0) to (4,0);
\draw[-,very thick] (-4,4) to (4,4);
\draw[-,very thick] (-4,-4) to (-4,4);
\draw[-,very thick] (0,-4) to (0,4);
\draw[-,very thick] (4,-4) to (4,4);
\draw[-,very thin] (-4,4) to (0,0);
\draw[-,very thin] (0,0) to (4,-4);
\draw[-,very thin] (-4,0) to (0,-4);
\draw[-,very thin] (0,4) to (4,0);
\end{tikzpicture}}

% Math sets
\newcommand{\N}{\mathbb{N}}
\newcommand{\Z}{\mathbb{Z}}
\newcommand{\R}{\mathbb{R}}

% Setup of project
\newenvironment{question}[2][Question]{\begin{trivlist}
\item[\hskip \labelsep {\bfseries #1}\hskip \labelsep {\bfseries #2.}]}{\end{trivlist}}
\newenvironment{answer}[2][Answer]{\begin{trivlist}
\item[\hskip \labelsep {\bfseries #1}\hskip \labelsep {\bfseries #2:}]}{\end{trivlist}}
\begin{document}

% Short hands
\let\oldsum\sum
\renewcommand{\sum}[3]{\oldsum\limits_{#1}^{#2}#3}
\let\oldprod\prod
\renewcommand{\prod}[3]{\oldprod\limits_{#1}^{#2}#3}

\title{Homework 5}
\author{Haukur Páll Jónsson\\
Game Theory}

\maketitle

\begin{question}{1}
Third price sealed-bid auctions
\end{question}
\begin{answer}{a)}{Workings}

This auction is a sealed-bid auction, i.e. a direct-revelation auction in which bidders either report a fake or true valuation via their bid. We assume that the auctioneer has a reservation price under which she won't sell, thus we need at least three bidders at or above the reservation price so that the auctioneer will sell. Each bidder $i$ has a private valuation $v_i$ of the item and places a bid of $\hat{v_i}$ for that item. The winner of the auction is the bidder with the highest bid, let us call that $\hat{v}^1$. She gets the item for the price of the 3rd highest bid, let us call that $\hat{v}^3$, thus her utility is defined as: $v_i-\hat{v}^3$ if she wins, $0$ otherwise.
\end{answer}
\begin{answer}{b)}{Strategy}

A good strategy for $i$ would be to bid $\hat{v}^1 + \epsilon$ as long as $\hat{v}^1 + \epsilon < v_i$, i.e. bid a little higher than the highest bid (assuming $i$ hasn't made a descision yet). If $i$ were to bid truthfully and $v_i=\hat{v}^2$ ($i$'s bid is the 2nd highest) then $i$ will lose and get utility $0$ thus she can increase her utility by bidding non-truthfully $\hat{v}^1 + \epsilon$ as long as $\hat{v}^1 + \epsilon < v_i$. Similiarly if $i$ bid were the highest then she could similarly bid $\hat{v}^2 + \epsilon$ and still win with the same utility, since the utility does not depend on our bid. If $v_i=\hat{v}^3$ or lower then it is impossible for $i$ to have a positive utility since if she were to bid $\hat{v}^1 + \epsilon$ then the price will go from being $\hat{v}^3$ to being $\hat{v}^2 > v_i$, by assumption. Thus, she will lose by doing this.
\end{answer}
\begin{answer}{c)}{Dominant strategies}

There is no dominant strategy for a third-price sealed-bid auctions. Consider the argument in b). In this argument we needed to consider what other players will do, thus our strategy depends on our idea of what they will bid. Similarly we can say that there is no dominant strategy since learning what the other bidders might do we might change our strategy from bidding $\hat{v}^1 + \epsilon$ to a losing bid.
\end{answer}
\begin{answer}{d)}{Pareto efficient}

It is not the case that the winner will always enjoy a non-negative utility. Consider the argument put forward in $b-c)$. If the probability $p=p(v_i\geq\hat{v}^2)$, i.e. the probability that $\hat{v}^1 > v_i \geq \hat{v}^2$, times the utility of winning plus $(1-p)=(v_i < \hat{v}^2)$ times the utility of losing is greater than $0$ then it would be rational for $i$ to bid greater than $v_i$. Thus in some cases $i$ might experience a negative utility but overall experience positive expected utility.
\end{answer}

\begin{question}{3}
A counter example to incentive-compatible claim
\end{question}
\begin{answer}{a)}

A mechanism is \textit{incentive-compatible} if truth telling is a dominant strategy. That is for all agents $i$, for all $v_i, \hat{v}_i \in V_i$ and $\boldsymbol{\hat{v}}_{-i} \in \boldsymbol{\hat{V}}_{-i}$:
$$v_i(f(v_i,\boldsymbol{\hat{v}}_{-i})) - p_i(v_i,\boldsymbol{\hat{v}}_{-i}) \geq v_i(f(\hat{v}_i,\boldsymbol{\hat{v}}_{-i})) - p_i(\hat{v}_i,\boldsymbol{\hat{v}}_{-i})$$

To prove that this mechanism is not incentive-compatible we need to provide an instance of this game in which it is better for some player $i$ to deviate from her true valuation, $v_i$.

Take a game with $n=3$ and $k=2$ and valuations $v_1=8$, $v_2=6$ and $v_3=4$ as a counter-example. Assuming truthful bidding player $1$ will win and pay $6$ thus having utility of $8-6=2$. Player two will be second, paying $4/2=2$ thus having utility $6-2=4$ and player $3$ will get nothing, utility $0$. But if player $1$ would deviate s.t. she plays $7$ instead, she would increase her utility because she would pay $6/2=3$ thus, having utility of $4-3=1$. Thus, there is a player which is better off by not playing her true valuation. Thus, this mechanism is not \textit{incentive-compatible}.
\end{answer}


\begin{question}{4}
Programming
\end{question}
\begin{answer}{a)}
I did this excersice. Please see the report.
\end{answer}

\end{document}