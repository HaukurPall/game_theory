\documentclass[12pt]{article}

\usepackage[margin=1in]{geometry}
\usepackage{amsmath,amsthm,amssymb}
\usepackage[utf8]{inputenc}
\usepackage[T1]{fontenc}
\usepackage{hyperref}

\newcommand{\N}{\mathbb{N}}
\newcommand{\Z}{\mathbb{Z}}

\newenvironment{question}[2][Question]{\begin{trivlist}
\item[\hskip \labelsep {\bfseries #1}\hskip \labelsep {\bfseries #2.}]}{\end{trivlist}}
\newenvironment{answer}[2][Answer]{\begin{trivlist}
\item[\hskip \labelsep {\bfseries #1}\hskip \labelsep {\bfseries #2:}]}{\end{trivlist}}
\begin{document}

\let\oldsum\sum
\renewcommand{\sum}[3]{\oldsum\limits_{#1}^{#2}#3}

\title{Homework 3}
\author{Haukur Páll Jónsson\\
Game Theory}

\maketitle

\begin{question}{1}
Constant sum game
\end{question}
\begin{answer}{a)}
Here we need to show that the set of strategy profiles which are in Nash equilibrium of a zero-sum game, call it $NE^z$, is the same as the set of strategy profiles which are in Nash equilibrium of a constant sum game, call it $NE^c$. So we need to show that:
$$NE^z=NE^c$$
So that for any strategy profile in $NE^z$ it is also in $NE^c$ as well as for any strategy profile in $NE^c$ it is also in $NE^z$.

Let us now assume a strategy profile $\boldsymbol{s^z} \in NE^z$. This implies that $\boldsymbol{s^z}$ is a Nash equilibrium, so in this strategy profile all players are playing their best response to the other players: for all players $i$ and all strategy profiles $\boldsymbol{s_i} \in \boldsymbol{S_i}$: $u^z_i(\boldsymbol{s^z_i}, \boldsymbol{s_{-i}}) \geq u^z_i(\boldsymbol{s_i}, \boldsymbol{s_{-i}})$. It is now clear that if we add $\frac{c}{2}$ on both sides of the inequality the inequality still holds, i.e. for all players $i$ and all strategy profiles $\boldsymbol{s_i} \in \boldsymbol{S_i}$: $u^z_i(\boldsymbol{s^z_i}, \boldsymbol{s_{-i}}) + \frac{c}{2} \geq u^z_i(\boldsymbol{s_i}, \boldsymbol{s_{-i}}) + \frac{c}{2}$ but this is clearly for all players $i$ and all strategy profiles $\boldsymbol{s_i} \in \boldsymbol{S_i}$: $u^c_i(\boldsymbol{s^z_i}, \boldsymbol{s_{-i}}) \geq u^c_i(\boldsymbol{s_i}, \boldsymbol{s_{-i}})$ since for all players $i$ $u^c_i(\boldsymbol{s})=u^z_i(\boldsymbol{s}) - \frac{c}{2}$. Since this is the definition of a Nash equilibrium of the constant sum game we can say that indeed $s^z \in NE^c$. The argument is exactly the same in the other direction, except that we subtract $\frac{c}{2}$ from $u^c_i$ instead.

Thus we have show that the set of strategy profiles in Nash equilibrium for a constant sum game is the same set as the set of strategy profiles in Nash equilibrium for a zero-sum game.
\end{answer}
\begin{answer}{b)}
We can create a strategy which will give different results in a constant game vs a zero-sum game. Take for an example the strategy in which a player only chooses actions which give more payoff than $x$, and chooses uniformly from that set of actions. If this set is empty let us say that the player chooses uniformly from all actions. Given a constant sum game in which the set of actions which have a higher payoff than $x$ is not empty and there are some actions in which are not in this set. If this game is then adjusted to a zero-sum game this player will not play the same actions.

This is a very trivial example but it seems to be sufficient.
\end{answer}
\begin{question}{2}
Programming. Simulating fictious game play
\end{question}
\begin{answer}{a)}
I did this programming assigment with Grzegorz.
See submission $jonsson\_liswiski.tar.gz$
\end{answer}

\begin{question}{3}
Bayesian game to normal form game
\end{question}
\begin{answer}{a)}
Let us define the types $knows-\alpha$ the player type which knows $\alpha$, $\alpha_0$ the player type which believes $\alpha$ to be $0$ and $\alpha_2$ similary to be $2$. Player $1$ is of type $knows-\alpha$ and player $2$ is with probability $0.5$ type $\alpha_0$ and with probability $0.5$ type $\alpha_2$. $\alpha$ is either $0$ or $2$. Let us say that with probability $p$ $\alpha$ is $0$ and $1-p$ $\alpha$ is $2$.

We now define the Bayesian game (N,)
\end{answer}

\end{document}