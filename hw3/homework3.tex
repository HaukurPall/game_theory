\documentclass[12pt]{article}

\usepackage[margin=1in]{geometry}
\usepackage{amsmath,amsthm,amssymb}
\usepackage{tikz} % for drawing stuff
\usepackage{xcolor} % for \textcolor{}
\usepackage{readarray} % for \getargsC{}
\usepackage[utf8]{inputenc}
\usepackage[T1]{fontenc}
\usepackage{hyperref}

% Colors for players
\definecolor{darkred}{rgb}{0.64,0,0}
\definecolor{darkcyan}{rgb}{0,0.55,0.55}
\newcommand{\rowcolor}[1]{\textcolor{darkred}{#1}}
\newcommand{\columncolor}[1]{\textcolor{darkcyan}{#1}}
% Normal-form game
% \nfgame{T B L R TLR TLC BLR BLR TRR TRC BRR BRC}
\newcommand{\nfgame}[1]{%
\getargsC{#1}
\begin{tikzpicture}[scale=0.65]
\node (RT) at (-2,1) [label=left:\rowcolor{\argi}] {};
\node (RB) at (-2,-1) [label=left:\rowcolor{\argii}] {};
\node (CL) at (-1,2) [label=above:\columncolor{\argiii}] {};
\node (CR) at (1,2) [label=above:\columncolor{\argiv}] {};
\node (RTL) at (-1.4,0.6) {\rowcolor{\argv}}; % top/left row player payoff
\node (CTL) at (-0.6,1.4) {\columncolor{\argvi}}; % top/left column player payoff
\node (RBL) at (-1.4,-1.4) {\rowcolor{\argvii}};
\node (CBL) at (-0.6,-0.6) {\columncolor{\argviii}};
\node (RTR) at (0.6,0.6) {\rowcolor{\argix}};
\node (CTR) at (1.4,1.4) {\columncolor{\argx}};
\node (RBR) at (0.6,-1.4) {\rowcolor{\argxi}};
\node (CBR) at (1.4,-0.6) {\columncolor{\argxii}};
\draw[-,very thick] (-2,-2) to (2,-2);
\draw[-,very thick] (-2,0) to (2,0);
\draw[-,very thick] (-2,2) to (2,2);
\draw[-,very thick] (-2,-2) to (-2,2);
\draw[-,very thick] (0,-2) to (0,2);
\draw[-,very thick] (2,-2) to (2,2);
\draw[-,very thin] (-2,2) to (0,0);
\draw[-,very thin] (0,0) to (2,-2);
\draw[-,very thin] (-2,0) to (0,-2);
\draw[-,very thin] (0,2) to (2,0);
\end{tikzpicture}}

% Math sets
\newcommand{\N}{\mathbb{N}}
\newcommand{\Z}{\mathbb{Z}}
\newcommand{\R}{\mathbb{R}}

% Setup of project
\newenvironment{question}[2][Question]{\begin{trivlist}
\item[\hskip \labelsep {\bfseries #1}\hskip \labelsep {\bfseries #2.}]}{\end{trivlist}}
\newenvironment{answer}[2][Answer]{\begin{trivlist}
\item[\hskip \labelsep {\bfseries #1}\hskip \labelsep {\bfseries #2:}]}{\end{trivlist}}
\begin{document}

% Short hands
\let\oldsum\sum
\renewcommand{\sum}[3]{\oldsum\limits_{#1}^{#2}#3}
\let\oldprod\prod
\renewcommand{\prod}[3]{\oldprod\limits_{#1}^{#2}#3}

\title{Homework 3}
\author{Haukur Páll Jónsson\\
Game Theory}

\maketitle

\begin{question}{1}
Constant sum game
\end{question}
\begin{answer}{a)}
We have two games: $G^c=(N,\boldsymbol{A}, \boldsymbol{u^c})$, a constant sum game with for all action profiles $u_i^c(\boldsymbol{a}) + u_{-i}^c(\boldsymbol{a})=c$, $c\in \R$ and $G^z=(N,\boldsymbol{A}, \boldsymbol{u^z})$, a zero-sum game with for all action profiles $u_i^c(\boldsymbol{a}) -u_{-i}^c(\boldsymbol{a})=0$. We can transform $G^z$ to $G^c$ by $u_i^z(\boldsymbol{a})=u_i^c(\boldsymbol{a}) - \frac{c}{2}$.

We need to show that the set of strategy profiles (possibly mixed) which are in Nash equilibrium of a zero-sum game, call it $NE^z$, is the same as the set of strategy profiles which are in Nash equilibrium of a constant sum game, call it $NE^c$. So we need to show that:
$$NE^z=NE^c$$
So that for any strategy profile if $\boldsymbol{s^z} \in NE^z$ then $\boldsymbol{s^z} \in NE^c$. Similarly, for any strategy profile if $\boldsymbol{s^c} \in NE^c$ then $\boldsymbol{s^c} \in NE^z$

If $\boldsymbol{s^z} \in NE^z$ then:
\begin{align*}
\forall i \in N, \forall \boldsymbol{s_i} \in \boldsymbol{S_i}: u^z_i(\boldsymbol{s^z_i}, \boldsymbol{s_{-i}}) &\geq u^z_i(\boldsymbol{s_i}, \boldsymbol{s_{-i}}) \\
\sum{\boldsymbol{a} \in \boldsymbol{A}}{}{u^z_i(\boldsymbol{a^z_i}, \boldsymbol{a_{-i}}) \cdot \prod{j \in N}{}{s_j(a_j)}} &\geq \sum{\boldsymbol{a} \in \boldsymbol{A}}{}{u^z_i(\boldsymbol{a_i}, \boldsymbol{a_{-i}}) \cdot \prod{j \in N}{}{s_j(a_j)}}\\
\sum{\boldsymbol{a} \in \boldsymbol{A}}{}{(u_i^c((\boldsymbol{a^z_i}, \boldsymbol{a_{-i}}) - \frac{c}{2} ) \cdot \prod{j \in N}{}{s_j(a_j)}} &\geq \sum{\boldsymbol{a} \in \boldsymbol{A}}{}{(u^c_i(\boldsymbol{a_i}, \boldsymbol{a_{-i}}) - \frac{c}{2}) \cdot \prod{j \in N}{}{s_j(a_j)}}\\
\sum{\boldsymbol{a} \in \boldsymbol{A}}{}{u_i^c(\boldsymbol{a^z_i}, \boldsymbol{a_{-i}}) \cdot \prod{j \in N}{}{s_j(a_j)}} - |A|\frac{c}{2}\cdot \prod{j \in N}{}{s_j(a_j)} &\geq \sum{\boldsymbol{a} \in \boldsymbol{A}}{}{u^c_i(\boldsymbol{a_i}, \boldsymbol{a_{-i}}) \cdot \prod{j \in N}{}{s_j(a_j)}} - |A|\frac{c}{2}\cdot \prod{j \in N}{}{s_j(a_j)} \\
\sum{\boldsymbol{a} \in \boldsymbol{A}}{}{u_i^c(\boldsymbol{a^z_i}, \boldsymbol{a_{-i}}) \cdot \prod{j \in N}{}{s_j(a_j)}} &\geq \sum{\boldsymbol{a} \in \boldsymbol{A}}{}{u^c_i(\boldsymbol{a_i}, \boldsymbol{a_{-i}}) \cdot \prod{j \in N}{}{s_j(a_j)}}\\
\end{align*}
This implies that $\boldsymbol{s^z} \in NE^c$ as well so we have shown that if $\boldsymbol{s^z} \in NE^z$ then $\boldsymbol{s^z} \in NE^c$. The argument is exactly the same in the other direction, except that we substitute $u_i^c(\boldsymbol{a})=u_i^z(\boldsymbol{a}) + \frac{c}{2}$ instead.

Thus we have show that the set of strategy profiles in Nash equilibrium for a constant sum game is the same set as the set of strategy profiles in Nash equilibrium for a zero-sum game.
\end{answer}
\begin{answer}{b)}
We can create a strategy which will give different results in a constant game vs a zero-sum game. Take for an example the strategy in which a player only chooses actions which give more payoff than $x$, and chooses uniformly from that set of actions. If this set is empty let us say that the player chooses uniformly from all actions. Given a constant sum game in which the set of actions which have a higher payoff than $x$ is not empty and there are some actions in which are not in this set. If this game is then adjusted to a zero-sum game this player will not play the same actions.

This is a very trivial example but it seems to be sufficient.
\end{answer}
\begin{question}{2}
Programming. Simulating fictitious game play
\end{question}
\begin{answer}{a)}
I did this programming assignment with Grzegorz.
See submission $jonsson\_liswiski.tar.gz$
\end{answer}

\begin{question}{3}
Bayesian game to normal form game
\end{question}
\begin{answer}{a)}
We have a Bayesian game $G=(N, \boldsymbol{A}, \boldsymbol{\Theta}, p, \boldsymbol{u})$ where $N=\{1,2\}$, $\boldsymbol{A}=A_1\times A_2$ where $A_1=\{T,B\}, A_2=\{L,R\}$, $\boldsymbol{\Theta}=\Theta_1\times\Theta_2$ where $\Theta_1=\{knows\}, \Theta_2=\{belives_{\alpha=0},belives_{\alpha=2}\}$. $p(knows, belives_{\alpha=0})=1/2$ and $p(knows, belives_{\alpha=2})=1/2$ and $u_1((T,L),(knows, belives_{\alpha=0}))=5$ etc., as described in the question. We need to add that if player 2 is of type $belives_{\alpha=0}$ then $u_2((T,R),(knows, belives_{\alpha=0}))=0$ and similarly if player 2 is of type $belives_{\alpha=2}$ then $u_2((T,R),(knows, belives_{\alpha=2}))=2$. That is, she will act as if those were her utilities. And $u_1((B,L),(knows, belives_{\alpha=0}))=\alpha=u_1((B,L),(knows, belives_{\alpha=2}))$

Thus we can translate G to a normal form game $G'=(N, \boldsymbol{A'}, \boldsymbol{u'})$ with $A_1=A_1'$ and $A_2=\{LL,LR,RL,RR\}$ and \textit{ex ante expected utility} s.t. $u_i'(\boldsymbol{s})=\sum{\boldsymbol{\theta} \in \boldsymbol{\Theta}}{}{u_i(\boldsymbol{s}, \boldsymbol{\theta}) \cdot p(\boldsymbol{\theta})}$. Thus we get this game:

\nfgame{T B LL LR RL RR $5$ $5$ $10$ $0$ $\alpha$ $10$ $1$ $1$ $5$ $5$ $10$ $2$ $\alpha$ $10$ $1$ $1$} \\[5pt]



Let us define the types $knows-\alpha$ the player type which knows $\alpha$, $\alpha_0$ the player type which believes $\alpha$ to be $0$ and $\alpha_2$ similary to be $2$. Player $1$ is of type $knows-\alpha$ and player $2$ is with probability $0.5$ type $\alpha_0$ and with probability $0.5$ type $\alpha_2$. $\alpha$ is either $0$ or $2$. Let us say that with probability $p$ $\alpha$ is $0$ and $1-p$ $\alpha$ is $2$.

We now define the Bayesian game (N,)
\end{answer}

\end{document}