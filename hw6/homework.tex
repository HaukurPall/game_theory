\documentclass[12pt]{article}

\usepackage[margin=1in]{geometry}
\usepackage{amsmath,amsthm,amssymb}
\usepackage{tikz} % for drawing stuff
\usepackage{xcolor} % for \textcolor{}
\usepackage{readarray} % for \getargsC{}
\usepackage{graphicx} % disjoint union
\usepackage[utf8]{inputenc}
\usepackage[T1]{fontenc}
\usepackage{hyperref}

% Colors for players
\definecolor{darkred}{rgb}{0.64,0,0}
\definecolor{darkcyan}{rgb}{0,0.55,0.55}
\newcommand{\rowcolor}[1]{\textcolor{darkred}{#1}}
\newcommand{\columncolor}[1]{\textcolor{darkcyan}{#1}}
% Normal-form game
% \nfgame{T B L R TLR TLC BLR BLR TRR TRC BRR BRC}
\newcommand{\nfgame}[1]{%
\getargsC{#1}
\begin{tikzpicture}[scale=0.65]
\node (RT) at (-2,1) [label=left:\rowcolor{\argi}] {};
\node (RB) at (-2,-1) [label=left:\rowcolor{\argii}] {};
\node (CL) at (-1,2) [label=above:\columncolor{\argiii}] {};
\node (CR) at (1,2) [label=above:\columncolor{\argiv}] {};
\node (RTL) at (-1.4,0.6) {\rowcolor{\argv}}; % top/left row player payoff
\node (CTL) at (-0.6,1.4) {\columncolor{\argvi}}; % top/left column player payoff
\node (RBL) at (-1.4,-1.4) {\rowcolor{\argvii}};
\node (CBL) at (-0.6,-0.6) {\columncolor{\argviii}};
\node (RTR) at (0.6,0.6) {\rowcolor{\argix}};
\node (CTR) at (1.4,1.4) {\columncolor{\argx}};
\node (RBR) at (0.6,-1.4) {\rowcolor{\argxi}};
\node (CBR) at (1.4,-0.6) {\columncolor{\argxii}};
\draw[-,very thick] (-2,-2) to (2,-2);
\draw[-,very thick] (-2,0) to (2,0);
\draw[-,very thick] (-2,2) to (2,2);
\draw[-,very thick] (-2,-2) to (-2,2);
\draw[-,very thick] (0,-2) to (0,2);
\draw[-,very thick] (2,-2) to (2,2);
\draw[-,very thin] (-2,2) to (0,0);
\draw[-,very thin] (0,0) to (2,-2);
\draw[-,very thin] (-2,0) to (0,-2);
\draw[-,very thin] (0,2) to (2,0);
\end{tikzpicture}}
% \nfgame{T B L R TLR TLC BLR BLR TRR TRC BRR BRC}
\newcommand{\nfgamebig}[1]{%
\getargsC{#1}
\begin{tikzpicture}[scale=0.65]
\node (RT) at (-4,1) [label=left:\rowcolor{\argi}] {};
\node (RB) at (-4,-1) [label=left:\rowcolor{\argii}] {};
\node (CL) at (-1,4) [label=above:\columncolor{\argiii}] {};
\node (CR) at (1,4) [label=above:\columncolor{\argiv}] {};
\node (RTL) at (-2.4,0.6) {\rowcolor{\argv}}; % top/left row player payoff
\node (CTL) at (-1.6,3.4) {\columncolor{\argvi}}; % top/left column player payoff
\node (RBL) at (-2.4,-3.4) {\rowcolor{\argvii}};
\node (CBL) at (-1.6,-0.6) {\columncolor{\argviii}};
\node (RTR) at (1.6,0.6) {\rowcolor{\argix}};
\node (CTR) at (2.3,3.4) {\columncolor{\argx}};
\node (RBR) at (1.6,-3.4) {\rowcolor{\argxi}};
\node (CBR) at (2.4,-0.6) {\columncolor{\argxii}};
\draw[-,very thick] (-4,-4) to (4,-4);
\draw[-,very thick] (-4,0) to (4,0);
\draw[-,very thick] (-4,4) to (4,4);
\draw[-,very thick] (-4,-4) to (-4,4);
\draw[-,very thick] (0,-4) to (0,4);
\draw[-,very thick] (4,-4) to (4,4);
\draw[-,very thin] (-4,4) to (0,0);
\draw[-,very thin] (0,0) to (4,-4);
\draw[-,very thin] (-4,0) to (0,-4);
\draw[-,very thin] (0,4) to (4,0);
\end{tikzpicture}}

% Math sets
\newcommand{\N}{\mathbb{N}}
\newcommand{\Z}{\mathbb{Z}}
\newcommand{\R}{\mathbb{R}}

% Setup of project
\newenvironment{question}[2][Question]{\begin{trivlist}
\item[\hskip \labelsep {\bfseries #1}\hskip \labelsep {\bfseries #2.}]}{\end{trivlist}}
\newenvironment{answer}[2][Answer]{\begin{trivlist}
\item[\hskip \labelsep {\bfseries #1}\hskip \labelsep {\bfseries #2:}]}{\end{trivlist}}
\begin{document}
% math enumerate
\renewcommand{\theenumi}{\roman{enumi}}

% Short hands
\let\oldsum\sum
\renewcommand{\sum}[3]{\oldsum\limits_{#1}^{#2}#3}
\let\oldprod\prod
\renewcommand{\prod}[3]{\oldprod\limits_{#1}^{#2}#3}

% Disjoint union
\newcommand\Dunion{
  \mathop{\mathchoice
    {\ooalign{$\displaystyle\bigcup$\cr\hss\scalebox{.65}{\raisebox{0.45ex}{\sffamily +}}\hss}}
    {\ooalign{$\textstyle\bigcup$\cr\hss\scalebox{.9}{\raisebox{0.5ex}{\tiny\sffamily +}}\hss}}
    {\ooalign{$\scriptstyle\bigcup$\cr\hss\scalebox{.45}{\raisebox{0.3ex}{\sffamily +}}\hss}}
    {\ooalign{$\scriptscriptstyle\bigcup$\cr\hss\scalebox{.38}{\raisebox{0.3ex}{\sffamily +}}\hss}}
    }
}

\title{Homework 6}
\author{Haukur Páll Jónsson\\
Game Theory}

\maketitle

\begin{question}{1}
TU game with which is balanced must be cohesive.
\end{question}
\begin{answer}{a)}{Proof}

Now we show that given a \textit{balanced} TU game we have a \textit{cohesive} TU game.

Balanced: $\sum{C \subseteq N}{}{\lambda_C \cdot v(C)} \leq v(N)$ with $\lambda_C \in [0,1]$ and $\sum{C \subseteq N}{}{\lambda_C}=1$

Cohesive: $N = C_1 \Dunion ... \Dunion C_k$ implies $v(N) \geq v(C_1) + ... + v(C_k)$

Assume $N' = C_1 \Dunion ... \Dunion C_k$ and $\sum{C \subseteq N'}{}{\lambda_C \cdot v(C)} \leq v(N')$

Thus given $N = C_1 \Dunion ... \Dunion C_k$ and $\sum{C \subseteq N}{}{\lambda_C \cdot v(C)} \leq v(N)$ we have $v(N) \geq v(C_1) + ... + v(C_k)$
\end{answer}

\begin{question}{2}
Compare Shapley axioms to Banzhaf axioms
\end{question}
\begin{answer}{a-d)}{Claim}

I claim that the \textit{Banzhaf value} does not satisfy the axiom of \textit{efficiency} but the axioms of \textit{symmetry}, \textit{dummy player} and \textit{additivity} are satisfied. The \textit{Banzhaf value}:
$$\beta_i(N,v)=\frac{1}{2^{n-1}}\cdot \sum{C \subseteq N \setminus \{i\}}{}{v(C \cup \{i\}) - v(C)}$$
\end{answer}

\begin{answer}{a)}{Efficiency}
Counter-example to \textit{efficiency}. \textit{Efficiency} is defined as: \\
$$\sum{i \in N}{}{x_i(N,v)=v(N)}$$
But consider the game presented on the slides: $N=\{1,2,3\}$ with $v(\{1\})=0$, $v(\{2\})=0$, $v(\{3\})=0$, $v(\{1,2\})=7$, $v(\{1,3\})=6$, $v(\{2,3\})=5$ and $v(N)=10$. Computing the \textit{Banzhaf value}: $\beta_1(N,v)=\frac{18}{4}$, $\beta_2(N,v)=\frac{16}{4}$ and $\beta_3(N,v)=\frac{14}{4}$. The sum is $\frac{48}{4}=12$ which is not equal to $v(N)=10$. Thus the \textit{Banzhaf value} is not \textit{efficient}.

\end{answer}
\begin{answer}{b)}{Symmetry}

Now we show that the \textit{Banzhaf value} satisfies the axiom of \textit{symmetry}:
$$\text{if }v(C \cup \{i\}) = v(C \cup \{j\}) \text{ for all } C \subseteq N \setminus \{i,j\} \text{, then } x_i(N,v)=x_j(N,v)$$
Let us assume that $v(C \cup \{i\}) = v(C \cup \{j\})$ for all $C \subseteq N \setminus \{i,j\}$.
Also note that |$C'$|=|$C''$| where $C' \subseteq N \setminus \{j\}$ and $C'' \subseteq N \setminus \{i\}$. Thus we get:
$$\sum{C \subseteq N \setminus \{i\}}{}{v(C \cup \{i\})}=\sum{C \subseteq N \setminus \{j\}}{}{v(C \cup \{j\})} \Leftrightarrow$$
$$\sum{C \subseteq N \setminus \{i\}}{}{v(C \cup \{i\})-v(C)}=\sum{C \subseteq N \setminus \{j\}}{}{v(C \cup \{j\})-v(C)} \Leftrightarrow$$
$$\frac{1}{2^{n-1}}\cdot\sum{C \subseteq N \setminus \{i\}}{}{v(C \cup \{i\})-v(C)}=\frac{1}{2^{n-1}}\cdot \sum{C \subseteq N \setminus \{j\}}{}{v(C \cup \{j\})-v(C)} \Leftrightarrow$$
$$\beta_i(N,v)=\beta_j(N,v)$$

\end{answer}
\begin{answer}{c)}{Dummy player}

Now we show that the \textit{Banzhaf value} satisfies the axiom of \textit{Dummy player}:
$$\text{if } v(C \cup \{i\})-v(C)=v(\{i\}) \text{, }C \subseteq N \setminus \{i\}\text{,}$$
$$\text{then } i \text{ is a dummy player s.t. } x_i(N,v)=v(\{i\})$$
We assume $v(C \cup \{i\})-v(C)=v(\{i\})$
$$\beta_i(N,v)=\frac{1}{2^{n-1}}\cdot \sum{C \subseteq N \setminus \{i\}}{}{v(C \cup \{i\}) - v(C)}=\frac{1}{2^{n-1}}\cdot \sum{C \subseteq N \setminus \{i\}}{}{v(\{i\})}$$
Since $v(\{i\})$ is a constant and |$C$|$=2^{n-1}$ where $C \subseteq N \setminus \{i\}$ thus
$$\beta_i(N,v)=\frac{2^{n-1}}{2^{n-1}}\cdot v(\{i\})=v(\{i\})$$

\end{answer}
\begin{answer}{d)}{Additivity}

Now we show that the \textit{Banzhaf value} satisfies the axiom of \textit{Additivity}:
$$x_i(N,v_1 + v_2)=x_i(N,v_1)+x_i(N,v_2)$$
$$\text{for } [v_1 + v_2]:C \mapsto v_1(C) + v_2(C)$$
\begin{align*}
\beta_i(N,v_1+v_2)&=\frac{1}{2^{n-1}}\cdot \sum{C \subseteq N \setminus \{i\}}{}{[v_1+v_2](C \cup \{i\}) - [v_1+v_2](C)} \\
&=\frac{1}{2^{n-1}}\cdot \sum{C \subseteq N \setminus \{i\}}{}{v_1(C \cup \{i\})+v_2(C \cup \{i\}) - v_1(C) -v_2(C)} \\
&=\frac{1}{2^{n-1}}\cdot \sum{C \subseteq N \setminus \{i\}}{}{v_1(C \cup \{i\}) - v_1(C)} + \frac{1}{2^{n-1}}\cdot \sum{C \subseteq N \setminus \{i\}}{}{v_2(C \cup \{i\}) - v_2(C)} \\
&=\beta_i(N,v_1)+\beta_i(N,v_2)
\end{align*}

\end{answer}
\begin{question}{3}
Programming - bankruptcy game
\end{question}
\begin{answer}{a)}{Proof}

I did this question with Greg. See report.
\end{answer}

\begin{question}{5}
Improved Gale-Shapley algorithm
\end{question}
\begin{answer}{a)}{Proof}

The algorithm terminates in a finite number of rounds. Consider the argument: There are finite many players. In each round a proposal is made. Each proposal can only be attempted once. For each player there are finitly many proposals available since there are finitly many players of the opposite sex, thus each player can only attempt finitly many times. Thus there are finite number of rounds.

The algorithm does not always end in matchings for all agents. Counter-example:
$$w_1:m_1 > m_2$$
$$w_2:m_1 > m_2$$
$$m_1:w_2 > w_1$$
$$m_2:w_1 > w_2$$
With sequence $(w_1,m_2,w_2,m_2)$

$w_1$ starts, proposes to $m_1$, is accepted. $m_2$ next, proposes to $w_1$, gets rejected. $w_2$ proposes to $m_1$, is accepted. Thus $w_1$ cannot propose to $m_2$ since she already rejected him and $m_2$ will propose to $w_2$ but she will reject him. Thus $m_2$ and $w_1$ are not eligible to propose anymore and are not engaged.

Now consider this counter-example to stability:
$$w_1:m_1 > m_3 > m_2$$
$$w_2:m_2 > m_1 > m_3$$
$$w_3:m_1 > m_3 > m_2$$
$$m_1:w_1 > w_2 > w_3$$
$$m_2:w_1 > w_2 > w_3$$
$$m_3:w_3 > w_1 > w_2$$
With sequence $(m_2,w_2,w_3,m_3,m_3,w_2,m_1,m_2,w_2)$

$m_2$ starts and proposes to $w_1$, is accepted. $w_2$ proposes to $m_2$, gets rejected. $w_3$ proposes to $m_1$, is accepted. $m_3$ proposes to $w_3$ and gets rejected. $m_3$ proposes to $w_1$, is accepted, $m_2$ and $w_1$ are now broken up. $w_2$ proposes to $m_1$, is accepted, $m_1$ and w_3$ are now broken up. $m_1$ propses to $w_1$, is accepted, $m_3$ and $w_1$ are now broken up. $m_2$ proposes to his only option, $w_3$ and is accepted. $w_2$ proposes to $m_3$, is accepted.

Thus we are left with $\{(m_1, w_1), (m_2, w_3), (m_3, w_2)\}$ but in fact $\{(m_1, w_1), (m_2, w_2), (m_3, w_3)\}$ would be more stable.

\end{answer}

\end{document}