\documentclass[12pt]{article}

\usepackage[margin=1in]{geometry}
\usepackage{amsmath,amsthm,amssymb}
\usepackage{tikz} % for drawing stuff
\usepackage{xcolor} % for \textcolor{}
\usepackage{readarray} % for \getargsC{}
\usepackage{graphicx} % disjoint union
\usepackage[utf8]{inputenc}
\usepackage[T1]{fontenc}
\usepackage{hyperref}

% Math sets
\newcommand{\N}{\mathbb{N}}
\newcommand{\Z}{\mathbb{Z}}
\newcommand{\R}{\mathbb{R}}

% Setup of project
\newenvironment{question}[2][Question]{\begin{trivlist}
\item[\hskip \labelsep {\bfseries #1}\hskip \labelsep {\bfseries #2.}]}{\end{trivlist}}
\newenvironment{answer}[2][Answer]{\begin{trivlist}
\item[\hskip \labelsep {\bfseries #1}\hskip \labelsep {\bfseries #2:}]}{\end{trivlist}}
\begin{document}
% math enumerate
\renewcommand{\theenumi}{\roman{enumi}}

% Short hands
\let\oldsum\sum
\renewcommand{\sum}[3]{\oldsum\limits_{#1}^{#2}#3}
\let\oldprod\prod
\renewcommand{\prod}[3]{\oldprod\limits_{#1}^{#2}#3}

% Disjoint union
\newcommand\Dunion{
  \mathop{\mathchoice
    {\ooalign{$\displaystyle\bigcup$\cr\hss\scalebox{.65}{\raisebox{0.45ex}{\sffamily +}}\hss}}
    {\ooalign{$\textstyle\bigcup$\cr\hss\scalebox{.9}{\raisebox{0.5ex}{\tiny\sffamily +}}\hss}}
    {\ooalign{$\scriptstyle\bigcup$\cr\hss\scalebox{.45}{\raisebox{0.3ex}{\sffamily +}}\hss}}
    {\ooalign{$\scriptscriptstyle\bigcup$\cr\hss\scalebox{.38}{\raisebox{0.3ex}{\sffamily +}}\hss}}
    }
}

\title{Homework 5}
\author{Haukur Páll Jónsson\\
Computational Social Choice}

\maketitle

\begin{question}{1}{Standard STV vs Elimination first STV}

Consider this example
\end{question}

\begin{answer}{a)}{}
Let $N=\{1,2,3,4,5\}$ be the set of voters, $X=\{a,b,c\}$ be the set of alternatives, let $k=2$ and the ballot given as: \\
$$a\succ_1 b\succ_1 c$$
$$a\succ_2 b\succ_2 c$$
$$a\succ_3 b\succ_3 c$$
$$a\succ_4 b\succ_4 c$$
$$c\succ_5 b\succ_5 a$$
First we observe that the $k$ winner STV rule defined in class will always elect $\{a,b\}$ but if we use the elimination first STV rule which is given in the homework then we eliminate $b$, the plurality loser, and get a winner set $\{a,c\}$. The crucial difference between the rules is that STV can favor the plurality loser.
\end{answer}

\begin{question}{2}{}


\end{question}
\begin{answer}{a)}{Borda with missing candidates in preferences}

I work under the assumption that when working with subsets of a ranking, some candidates will be missing from the ordering but other than that, they induce a strict ordering over the subset. I, therefore, do not consider that some candidates might be equivalent in the order or that the ordering is somehow broken. Every voter should assign equally many points, $m \cdot (m-1)/2$. Therefore, we might assume that candidates which are not ordered are the ones which are least preferred. We then assign $m-1$ points to the most preferred candidate, $m-2$ to the next and so on. When we reach the end of the ordering, we distribute the remaining points equally between the unranked candidates.

\end{answer}
\begin{answer}{b)}{May's theorem not automatable}

When trying to find an answer to this question I found the paper "Proving classical theorems of social choice theory in modal logic" by Giovanni Ciná and Ulle Endriss from 2016. In this paper it is mentioned that it is an open question wether May's theorem could be proven in this logic automatically. The proof method used in this paper is a one which translates a theorem of the logic used in the paper to a theorem of propositional logic and then checks if there exists a model which satisfies that formula. The problem one might encounter when attempting to prove May's theorem with this approach might be that the number of propositional variables needed to express the theorem is simply too great (since the satisfiability problem is NP-complete) and therefore not computable in sufficient amount of time. A satisfiability tool for the logic used in the paper simply does not exists and creating such a tool and getting it to the level of todays SAT solvers takes a while to do. This is likely to be a problem for logics tailored for a specific problem and therefore most of those approach might try to translate theorems to propositional logic with the above mentioned consequence.
\end{answer}
\begin{answer}{c)}{Single result Chamberlin and Courant}

If Chamberlin and Courant rule is used with $k=1$ then it corresponds with the Borda rule. If we assign regret to all candidates of each voter, the most preferred option having no regret. The rule elects the candidate with this minimal score. This score is the reversed Borda score, if we reverse the score, to the Borda score, and elect the candidate with the highest score we have the Borda rule.
\end{answer}
\begin{answer}{d)}{Slater in JA}

Let $\Phi$ be the set of agenda. We consider a profile, $\boldsymbol{J}$, of complete and consistent judgment sets, $\mathcal{J}(\Phi)$. We can translate the Slater rule to JA like so: The Slater rule elects $j \in \mathcal{J}(\Phi)$ which minimises $\sum{k \in \boldsymbol{J}}{}{H(j,k)}$, where $H(j,k)$ is the Hamming-distance between $j$ and $k$.

\end{answer}
\begin{answer}{e)}{Impartial culture in JA}

Impartial culture in JA would translate like so: Given a set of agenda, $\Phi$ then each complete and consistent judgment sets, $\mathcal{J}(\Phi)$ is equally likely to be held by a voter, i.e. a uniform probability.
\end{answer}
\end{document}