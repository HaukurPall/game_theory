\documentclass[12pt]{article}

\usepackage[margin=1in]{geometry}
\usepackage{amsmath,amsthm,amssymb}
\usepackage{tikz} % for drawing stuff
\usepackage{xcolor} % for \textcolor{}
\usepackage{readarray} % for \getargsC{}
\usepackage{graphicx} % disjoint union
\usepackage[utf8]{inputenc}
\usepackage[T1]{fontenc}
\usepackage{hyperref}

% Math sets
\newcommand{\N}{\mathbb{N}}
\newcommand{\Z}{\mathbb{Z}}
\newcommand{\R}{\mathbb{R}}

% Setup of project
\newenvironment{question}[2][Question]{\begin{trivlist}
\item[\hskip \labelsep {\bfseries #1}\hskip \labelsep {\bfseries #2.}]}{\end{trivlist}}
\newenvironment{answer}[2][Answer]{\begin{trivlist}
\item[\hskip \labelsep {\bfseries #1}\hskip \labelsep {\bfseries #2:}]}{\end{trivlist}}
\begin{document}
% math enumerate
\renewcommand{\theenumi}{\roman{enumi}}

% Short hands
\let\oldsum\sum
\renewcommand{\sum}[3]{\oldsum\limits_{#1}^{#2}#3}
\let\oldprod\prod
\renewcommand{\prod}[3]{\oldprod\limits_{#1}^{#2}#3}

% Disjoint union
\newcommand\Dunion{
  \mathop{\mathchoice
    {\ooalign{$\displaystyle\bigcup$\cr\hss\scalebox{.65}{\raisebox{0.45ex}{\sffamily +}}\hss}}
    {\ooalign{$\textstyle\bigcup$\cr\hss\scalebox{.9}{\raisebox{0.5ex}{\tiny\sffamily +}}\hss}}
    {\ooalign{$\scriptstyle\bigcup$\cr\hss\scalebox{.45}{\raisebox{0.3ex}{\sffamily +}}\hss}}
    {\ooalign{$\scriptscriptstyle\bigcup$\cr\hss\scalebox{.38}{\raisebox{0.3ex}{\sffamily +}}\hss}}
    }
}

\title{Homework 2}
\author{Haukur Páll Jónsson\\
Computational Social Choice}

\maketitle

\begin{question}{1}{Anti-plurality}


\end{question}

\begin{answer}{a)}{Consensus}

Claim: \textit{Unanimous ranking} consensus criterion and \textit{discrete distance} characterize the anti-plurality rule.


\end{answer}

\begin{answer}{b)}{Distance}
Claim: \textit{Majority winner} consensus criterion and \textit{swap distance} characterize the anti-plurality rule.

\end{answer}

\begin{answer}{c)}{Noise model}


\end{answer}

\begin{question}{2}{Anonymous and neutral}

\end{question}

\begin{answer}{a)}{Proof}
Claim: For $|X| = 3$ and $n$ voters and a voting rule which is \textit{anonymous} and \textit{neutral} it is not possible to get a three-way tie if $n\pmod 3=1$ or $n\pmod 3=2$.

Let us first show that it is impossible to have a voting rule which satisfies anonymity and neutrality with $|X|=3$. Consider the \textit{condorcet paradox}:

\begin{align*}
a \succ b \succ c \\
c \succ a \succ b \\
b \succ c \succ a \\
\end{align*}
Assume w.l.o.g. that $a$ wins. Then by neutrality we exchange $a$ and $b$ and then $b$ and $c$:
\begin{align*}
c \succ a \succ b \\
b \succ c \succ a \\
a \succ b \succ c \\
\end{align*}
Then the winner should be $c$ but if we apply neutrality then we get:
\begin{align*}
a \succ b \succ c \\
c \succ a \succ b \\
b \succ c \succ a \\
\end{align*}
This is again our original case, so $a$ should win. Contradiction. Since $a$ was chosen arbitrarily the argument applies to all resolute voting rules. The same argument applies if we assume w.l.o.g. that there is a tie between $a$ and $b$. Thus, the only possible outcome is a three-way tie between $a$, $b$ and $c$ (no alternative is not an option).

Thus we can construct the same argument with $x \in \N$ voters for each preference order in the condorcet paradox, $3 \cdot x$ voters in total. Thus, for voters of any multiple of $3$, there is no voting rule which satisfies anonymity and neutrality with $|X|=3$ which does not lead to a three-way tie.

Claim: The \textit{plurality rule} satisfies anonymity and neutrality never returns a three-way tie for $n\pmod 3=1$ or $n\pmod 3=2$ voters for $|X|=3$.

It is easy to see that the plurality rule satisfies anonymity and neutrality. Let us now show that it is impossible to have a three-way tie with the plurality rule with $n\pmod 3=1$ or $n\pmod 3=2$ voters.

Consider the total number of points available for the plurality rule for $n$ voters. It is $n$. If there is a three-way tie for $|X|=3$ the plurality rule it implies that those points are divided equally between all alternatives. Since only whole numbers are considered this can only be the case if $n\pmod 3=0$. In all other cases a three-way tie is impossible.
\end{answer}

\begin{question}{3}{Combinatorics}

For $N=\{1,2, ..., n\}$ voters and $X=\{x_1, x_2, ..., x_m\}$ alternatives we count the number of social choice functions, $\mathcal{L}(X)^n \mapsto 2^X$, possible given certain constraints.
\end{question}
\begin{answer}{a)}{Total}

Cardinality of the domain $\mathcal{L}(X)^n$:

For each ballot we can order the $m$ alternatives, thus there are $m!$ combinations for each ballot.
We have $n$ independent ballots, thus we have $m!^n$ combinations of ballots.
We order the ballots to achieve complete profiles, thus $m!^n \cdot n!$ combinations of profiles. In these combinations there are profiles which ''look the same,, (for example, they contain exactly the same preference orders in order) but since we consider each ballot as unique, we count each of them.

Cardinality of the codomain $2^X$:

We consider resolute and non-resolute social choice functions. Thus the possbile outcomes of a function are:

$$\binom{m}{1} + \binom{m}{2} + ... + \binom{m}{m}$$

We do not need to order outcomes.

Then the number of functions, given the cardinality of the domain and range (in our case the codomain is also the range), is:
\begin{align*}
|(2^X)|^{|\mathcal{L}(X)^n|}& \\
&=($$\binom{m}{1} + \binom{m}{2} + ... + \binom{m}{m})^{m!^n \cdot n!} \\
&=($$\binom{3}{1} + \binom{3}{2} + \binom{3}{3})^{3!^3 \cdot 3!} \\
&=($$3 + 3 + 1)^{3!^4}= 7^{6^4}=7^1269
\end{align*}


\end{answer}
\begin{answer}{b)}{Anonymous}

Since
\end{answer}
\begin{answer}{c)}{Neutral}


\end{answer}
\begin{answer}{d)}{Anonymous and Neutral}


\end{answer}
\begin{answer}{e)}{Resolute and Anonymous}


\end{answer}
\begin{answer}{f)}{Resolute and Neutral}


\end{answer}
\begin{answer}{g)}{Resolute, Anonymous and Neutral}


\end{answer}
\end{document}