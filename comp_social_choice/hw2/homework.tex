\documentclass[12pt]{article}

\usepackage[margin=1in]{geometry}
\usepackage{amsmath,amsthm,amssymb}
\usepackage{tikz} % for drawing stuff
\usepackage{xcolor} % for \textcolor{}
\usepackage{readarray} % for \getargsC{}
\usepackage{graphicx} % disjoint union
\usepackage[utf8]{inputenc}
\usepackage[T1]{fontenc}
\usepackage{hyperref}

% Math sets
\newcommand{\N}{\mathbb{N}}
\newcommand{\Z}{\mathbb{Z}}
\newcommand{\R}{\mathbb{R}}

% Setup of project
\newenvironment{question}[2][Question]{\begin{trivlist}
\item[\hskip \labelsep {\bfseries #1}\hskip \labelsep {\bfseries #2.}]}{\end{trivlist}}
\newenvironment{answer}[2][Answer]{\begin{trivlist}
\item[\hskip \labelsep {\bfseries #1}\hskip \labelsep {\bfseries #2:}]}{\end{trivlist}}
\begin{document}
% math enumerate
\renewcommand{\theenumi}{\roman{enumi}}

% Short hands
\let\oldsum\sum
\renewcommand{\sum}[3]{\oldsum\limits_{#1}^{#2}#3}
\let\oldprod\prod
\renewcommand{\prod}[3]{\oldprod\limits_{#1}^{#2}#3}

% Disjoint union
\newcommand\Dunion{
  \mathop{\mathchoice
    {\ooalign{$\displaystyle\bigcup$\cr\hss\scalebox{.65}{\raisebox{0.45ex}{\sffamily +}}\hss}}
    {\ooalign{$\textstyle\bigcup$\cr\hss\scalebox{.9}{\raisebox{0.5ex}{\tiny\sffamily +}}\hss}}
    {\ooalign{$\scriptstyle\bigcup$\cr\hss\scalebox{.45}{\raisebox{0.3ex}{\sffamily +}}\hss}}
    {\ooalign{$\scriptscriptstyle\bigcup$\cr\hss\scalebox{.38}{\raisebox{0.3ex}{\sffamily +}}\hss}}
    }
}

\title{Homework 2}
\author{Haukur Páll Jónsson\\
Computational Social Choice}

\maketitle

\begin{question}{1}{Anti-plurality}

By the anti-plurality rule, the candidate is voted the winner which is ranked last least as often, i.e. the best candidate is most seldom ranked last. Thus we can understand that a consensus can be reached if a majority agrees on which alternative is ranked last, second last, etc.. This clearly implies that there is a majority which agrees on which candidates to rank the highest.
\end{question}

\begin{answer}{a)}{Consensus}

Majority... not sure.
\end{answer}

\begin{answer}{b)}{Distance}


\end{answer}

\begin{answer}{c)}{Noise model}


\end{answer}

\begin{question}{2}{Anonymous and neutral}

\end{question}

\begin{answer}{a)}{Proof}

What is a three-way tie? It is the case when a voting rule F returns $X$ again. (for $x=3$)

Claim: For $|X| = 3$ and $n$ voters and a voting rule which is anonymous and neutral it is not possible to get a three-way tie with a if $\mod{n}{3}=1$ or $\mod{n}{3}=2$.

We say that a voting rule $F$ returns a three-way tie if $F(X)=X$, for $|X| = 3$

Start by showing that it is possible for $\mod{n}{3}=0$. Construct the profiles in such a way that
\end{answer}

\begin{question}{3}{}

\end{question}
\end{document}