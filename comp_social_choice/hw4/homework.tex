\documentclass[12pt]{article}

\usepackage[margin=1in]{geometry}
\usepackage{amsmath,amsthm,amssymb}
\usepackage{tikz} % for drawing stuff
\usepackage{xcolor} % for \textcolor{}
\usepackage{readarray} % for \getargsC{}
\usepackage{graphicx} % disjoint union
\usepackage[utf8]{inputenc}
\usepackage[T1]{fontenc}
\usepackage{hyperref}

% Math sets
\newcommand{\N}{\mathbb{N}}
\newcommand{\Z}{\mathbb{Z}}
\newcommand{\R}{\mathbb{R}}

% Setup of project
\newenvironment{question}[2][Question]{\begin{trivlist}
\item[\hskip \labelsep {\bfseries #1}\hskip \labelsep {\bfseries #2.}]}{\end{trivlist}}
\newenvironment{answer}[2][Answer]{\begin{trivlist}
\item[\hskip \labelsep {\bfseries #1}\hskip \labelsep {\bfseries #2:}]}{\end{trivlist}}
\begin{document}
% math enumerate
\renewcommand{\theenumi}{\roman{enumi}}

% Short hands
\let\oldsum\sum
\renewcommand{\sum}[3]{\oldsum\limits_{#1}^{#2}#3}
\let\oldprod\prod
\renewcommand{\prod}[3]{\oldprod\limits_{#1}^{#2}#3}

% Disjoint union
\newcommand\Dunion{
  \mathop{\mathchoice
    {\ooalign{$\displaystyle\bigcup$\cr\hss\scalebox{.65}{\raisebox{0.45ex}{\sffamily +}}\hss}}
    {\ooalign{$\textstyle\bigcup$\cr\hss\scalebox{.9}{\raisebox{0.5ex}{\tiny\sffamily +}}\hss}}
    {\ooalign{$\scriptstyle\bigcup$\cr\hss\scalebox{.45}{\raisebox{0.3ex}{\sffamily +}}\hss}}
    {\ooalign{$\scriptscriptstyle\bigcup$\cr\hss\scalebox{.38}{\raisebox{0.3ex}{\sffamily +}}\hss}}
    }
}

\title{Homework 4}
\author{Haukur Páll Jónsson\\
Computational Social Choice}

\maketitle

\begin{question}{1}{Necessary winner}

Let $N=\{1,2, ..., n\}$ be the set of voters, $X=\{x_1,x_2,...,x_m\}$ be the set of candidates.
I provide a polynomial algorithm which produeces the \textit{Necessary winner} for the plurality rule.
\end{question}

\begin{answer}{a)}{The algorithm}

We assume w.l.o.g. that we know $|N|=n$ and $|X|=m$.

To begin with we define and store the \textit{maximal score of a candidate $x \in X$} as $max(x)=n$. Similarly we define and store the \textit{minimal score of a candidate $x \in X$} as $min(x)=0$. This takes $O(m)$ steps.

Furthermore, for each $i \in N$ we define and store the set of \textit{undominated candidates for voter $i \in N$} as $undom(i)$, to begin with the set contains all candidates. This takes $O(n)$ steps.

We assume that the preferences are \textit{elicit finely}, that is, we assume that each information increment consists of $x \succ_i y$. There are in total $n \cdot m \cdot (m-1)$ such increments. For every given $x \succ_i y$ we remove $y$, if present
\footnote{Many of the information increments do not contain information which will help deciding the necessary winner. This is because we do not care about the complete order, we only care about the cases when we remove a candidate from the set of \textit{undominated candidates for voter i} and it is possible that we already seen $x \succ_j y$ which implies that we have removed $y$ from $undom(j)$ and $y \succ_j z$, remove $z$ from $undom(j)$ and therefore $x \succ_j z$ would not give us more information since we cannot remove $z$ again from $undom(j)$.}
, from $undom(i)$.

If $y$ was removed from $undom(i)$ then we decrement $max(y)$ by 1. Furthermore, if $x$ is now the only element of $undom(i)$ then we increment $min(x)$ by 1.

If $y$ was removed from $undom(i)$, then we then define a \textit{possible winner} as candidate $x^*$ which satisfy $\max_{x \in X} min(x) \leq max(x^*)$. Checking this condition takes $O(m)$ steps. If there is only one \textit{possible winner} she is the \textit{necessary winner}. If there are multiple \textit{possible winners}, then we define a \textit{necessary winner} as all possible winners $x^*$ if for all other possible winners $x'$: $max(x^*) = max(x') =min(x^*) = min(x')$, i.e. they are tied in first postion and cannot get more points. Checking this condition takes $O(m)$ steps.

We therefore end up, roughly, with the running time of: $O(n) + O(m) + O((n \cdot m^2) \cdot (O(m) + O(m)))=O(n \cdot m^3)$\footnote{How should you use the Big-oh notation to argue algebraically? Is this ok?}

If we would use \textit{course elicitation} the algorithm would run would be faster since then in each information increment we would remove $m-1$ candidates from $undom(i)$, leaving only one, in each iteration. The running time would be, roughly, $O(n \cdot m^2)$.
\end{answer}

\begin{question}{2}{Iterative voting}

Let $N=\{1,2, ..., n\}$ be the set of voters, $X=\{x_1,x_2,x_3\}$ be the set of candidates. $F$ be the \textit{borda rule}. $\boldsymbol{R^0}$ be \textit{true preferences}. We only consider \textit{best responses} and break ties using a fixed \textit{lexicographic order} and show that iterated voting does need not converge.
\end{question}

\begin{answer}{a)}{Counterexample}

$$\succ_1=a\succ c\succ b$$
$$\succ_2=c\succ b\succ a$$
We start with $\boldsymbol{R^0}=(\succ_1,\succ_2)$, i.e. $[2,1,\textcolor{red}{3}]$.

Voter 1, puts $b$ up and $c$ down (by argument of \textit{best responses}) and votes:
$$\succ_1=a\succ b\succ c$$
$$[\textcolor{red}{2},2,2]$$

Voter 2, puts $b$ up and $c$ down we get:
$$\succ_2=b\succ c\succ a$$
$$[2,\textcolor{red}{3},1]$$

Voter 1, puts $b$ back down again and $c$ up again we get:
$$\succ_1=a\succ c\succ b$$
$$[\textcolor{red}{2},2,2]$$

Voter 2, also falls back to her truthful preference:
$$\succ_2=c\succ b\succ a$$
$$[2,1,\textcolor{red}{3}]$$

We are now in the exact same position as to begin with and voter 1 has again an incentive to put $b$ down and $c$ up.
\end{answer}

\begin{question}{3}{Combinatorial domains}

Using the language of prioritised goals and restrict ourselves s.t. voters can only specify a single goal.

Let $\mathcal{D}=D_1 \times D_2 \times ... \times D_p$ with $|D_i|=2$ and we assciate $\mathcal{D}$ with a set $PS=\{X_1, ..., X_p\}$ of propositional variables.
\end{question}

\begin{answer}{a)}{Weak condorcet winner}

First I show that that a \textit{weak condorcet winner} need not be unique. Consider this example with only a single voter, two alternatives, $x$ and $y$, and a goal $x \lor y:1$. This induces this preference order:

$$xy \sim \overline{x}y  \sim x\overline{y} \succ_1 \overline{x}\overline{y}$$

In which there are three \textit{weak condorcet winners}.

Now I show that a \textit{weak condorcet winner} always exists for the case when voters only express a single goal, $\phi$.

We start by observing that restricting our selves to a single goal representation, each voter will induce at most two equivalence classes on the set of alternatives since the equivalence classes are defined on when an alternative satisfies a goal or not. For a single goal, it is either satisfied or not. In the case of tautologies or contradictions only one equivalence classe is induced but in those cases each alternative ties with another, thus fulfulling the requirmment of a \textit{weak condorcet winner}.

Each alternative is either satisfied by a voter's goal and it always the case that for each voter there is an alternative which more prefered than another, $\exists x:x \succ_i y$, for some voter $i$ or they are all equivalent, $\forall x,x':x \neq x' \land x \sim x'$

Now consider the construction of a profile by adding preference orders which should eliminate\footnote{The construction is general but the incentive is that it should elimiate \textit{weak condorcet winners}}\textit{weak condorcet winners} and show that it always contains a \textit{weak condorcet winner}. We have already showed that there is always an alternative with is more prefered than other alternatives or they are all equivalent thus we choose w.l.o.g. a goal for our first voter which is only satisfied by 1 alternative, $x^1$. At this stage $x^1$ is the \textit{condorcet winner} (then also the weak). For the second voter we choose another goal which is only satisfied by 1 alternative. That alternative can either be $x^1$ or $x^2$, distinct from $x^1$. If we choose $x^1$ then $x^1$ continues to be a \textit{condorcet winner}. If we choose $x^2$, distinct from $x^1$ then $x^1$ and x^2$ are both \textit{weak condorcet winners}. For the third voter we can, depending on what we did for voter two, do one of the following. Choose to satisfy an already established \textit{condorcet winner}, a \textit{weak condorcet winner} making it a \textit{condorcet winner} or an alternative which is neither an making all of them \textit{weak condorcet winners}. This construction can continue until we have exhausted all candidates and we are required to choose a previous \textit{weak condorcet winner} making it a \textit{condorcet winner}.

If we now consider a construction in which we choose to satisfy multiple candidates at once, then we are essentially doing this selection twice in a row. One might object and say that it is possible to choose an alternative which is neither a \textit{condorcet winner} nor a \textit{weak condorcet winner} but will not make that candidate a \textit{weak condorcet winner}, which is true, but it implies that previously you have chosen to satisfy other alternative(s) more than that one, making them the \textit{condorcet winner} or \textit{weak condorcet winner}.

Therefore, no matter how we decide to construct the profile, at each point there will be a (or multiple) \textit{weak condorcet winner}. Therefore, there always exists a \textit{weak condorcet winner} if each voter only specifies only a single goal.
\end{answer}

\end{document}