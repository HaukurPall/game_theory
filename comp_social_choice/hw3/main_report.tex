\documentclass[10pt,a4paper]{article}
\usepackage[utf8]{inputenc}

\title{%
  COMSOC HW 3: Report on Question 5 \\
  \large Max Rapp, Haukur J{\'o}nsson, Silvan Hungerb{\"u}hler}

\usepackage{mathptmx} % "times new roman"
\usepackage{amssymb}
\usepackage{amsmath, amsthm}
\usepackage{amsfonts}
\usepackage{enumitem}
\usepackage{verbatim}
\usepackage{hyperref}
\usepackage{comment}
%\usepackage[margin=1in]{geometry}
\usepackage{float}
\usepackage{bm}
\usepackage{graphicx}
\graphicspath{ {/home/sh/Desktop/CSC/CSC_HW/miniproject/images/} }
\usepackage[rightcaption]{sidecap}
\usepackage[normalem]{ulem}
\date{}
\begin{document}
\maketitle
\section{Introduction}
It is clear from the Gibbard-Satterthwaite Theorem that any reasonable voting rule is in principle prone to manipulation. That is, there will be profiles for which some voters stand to improve the outcome of the vote by reporting dishonest preferences. It is unclear, however, how often such situations will arise in practice. We are therefore asked to assess and compare the \textit{Borda}, \textit{Plurality} and \textit{Copeland} rules with respect to their practical proneness to manipulability.

There are various factors that could potentially influence the prevalence of manipulability: the number of candidates partaking in the election, the voting rule employed, the number of voters in the profile and assumptions concerning the distribution over possible profiles. Due to time constraints we are unable to tinker with the last parameter, so we have not analyzed it. We simply use the impartial culture assumption throughout this report. In order to individually account for the remaining three factors we set up a logistic regression model. This allows us to separately account for each factor's effect on the expected probability of a given profile-cum-voting rule being manipulable. Moreover, it provides us with confidence intervals for the effects we observe; that is, we obtain some measure of how robust our results are. 

Using algorithms described in Section 2 we generated a data set in the following way:
First, we randomly generated $100$ profiles for all either $6$ or $20$ candidates and between between $3$ and $1000$ voters.  Secondly, we determined the winner in all of the profiles under each voting rule. Then we checked for each profile and voting rule whether some voter in the profile under consideration a) stands to gain from successful manipulation and b) has the powers to induce it. If both points are satisfied, then we declare the outcome manipulable, otherwise, we declare it non-manipulable. For each profile-voting rule pair we thus get a binary verdict: manipulable or not. The fraction of manipulable profiles under some voting rule is the voting rule's practical proneness to manipulation.

With the data at hand we first give a descriptive overview over the respective proneness of each voting rule for different numbers of voters. Specifically, we consider two types of elections: one with $3$ and one with $6$ candidates. For each of the two types we compute the fraction of profiles that are manipulable for $100$ profiles with between $10$ and $1000$ voters (in intervals of $10$). Ensuingly, we undertake a logistic regression analysis where we try to disentangle which of the three factors we identified contributes to which degree to likely manipulability. The main focus of our analysis at this stage, however, still lies on the effect the number of voters has on the manipulability of a given voting rule.

The report is structured as follows: The second Section provides some detail on the algorithms we used to determine the winners for a given profile-voting rule pair and whether said outcome can be manipulated. The third Section presents a descriptive analysis of the data we generated as well as a logistic regression analysis to account for different factors separately. The fourth Section discusses the findings from Section 3 and concludes the report.
\section{Determining Winners \& Manipulability Under Each Voting Rule}
This Section sketches the algorithms we used to establish which candidate wins an election under a given voting rule for some profile and whether manipulability is an issue.
\subsection{Determining Winners}
Given some profile $\mathbf{R}$ we determined the winner by assigning each candidate a counter on which we put points. The candidate with most points wins, ties are broken lexicographically. The points were awarded thus for the three rules:
\begin{itemize}
\item \textit{Plurality}: A candidate's points is the number of times the candidate shows up in the first spot.
\item \textit{Borda}: The points of a candidate are her Borda-score.
\item \textit{Copeland}: A candidate's points are the number of pair-wise matchups a candidate wins minus the number of pair-wise matchups a candidate loses. (We view this rule effectively as a positional scoring rule. The scoring vector depends on the number of candidates - $|X|=x$ - and looks as follows: $s=(x-1, x-3, x-5, ..., -(x-3), -(x-1))$.)
\end{itemize}

\subsection{Determinining Manipulability}
\noindent The algorithms to determine the manipulability run as follows:\\
\begin{itemize}
\item \textit{Plurality}: \begin{enumerate}
\item  Compute the respective margin of victory (MOV) between \textit{actual winner} and all other candidates. If MOV$>1$ for all candidates, then declare the profile-rule pair non-manipulable. (One invdividual voter whose first option is not the winner can at most 'push' another candidate by $1$ point. If the MOV is larger than this she can not manipulate successfully.) Put all the candidates for whom MOV$<2$ in a set of \textit{potential winners}. These are the candidates in 'striking distance' who could still be manipulated into winning.
\item For all candidates in \textit{potential winners} look at each voter's profile and see if two conditions hold: \\
a) The \textit{potential winner} is strictly preferred to the \textit{actual winner} by the voter under consideration.\\
b) The \textit{potential winner} was not already ranked in the first position by the voter.\\
If both conditions hold for some voter, then declare manipulability. Otherwise, declare non-manipulability.
\end{enumerate}
\item \textit{Borda}: \begin{enumerate}
\item Compute the MOV for between all candidates and the \textit{actual winner}. If MOV$>x-2$ for all candidates, then declare the profile-rule pair non-manipulable. (One invdividual voter whose first option is not the winner can at most 'push' another candidate by $x-2$ point. If the MOV is larger than this she can not manipulate successfully.) Put all the candidates for whom MOV$<x-1$ in a set of \textit{potential winners}. These are the candidates in 'striking distance'.
\item For all candidates in \textit{potential winners} look at each voter's profile and see if three conditions hold: \\
a) The \textit{potential winner} is strictly preferred to the \textit{actual winner} by the voter under consideration.\\
b) The \textit{potential winner} was not already ranked in the first position by the voter.\\
If some condition fails to hold, move to next candidate. Otherwise, proceed in the following way:\\
Move the \textit{potential winner} all the way to the top of the ballot under consideration. Put all the hopeless candidates (those who neither win nor are part of the \textit{potential winners} right below the candidate we are currently trying to promote. At the bottom of the ranking try all possible permutations of ranking the remaining candidates. For each possible permutation compute the winner. If the \textit{potential winner} manages to win instead of the \textit{actual winner}, declare the profile manipulable. If for all candidates and all voters thus considered such a permutation cannot be found, declare non-manipulability.\\
Perhaps by way of clarification, we cannot simply put the \textit{actual winner} all the way to the bottom when checking for manipulability but need to proceed by considering the permutations one-by-one because of situations like the following:\\
Assume there are five candidates. The three candidates at the top of the Borda-score ranking - $A,B,C$ - have this many points ($\alpha$ being some positive integer):\\
$A$: $\alpha$\\
$B$: $\alpha-2$\\
$C$: $\alpha-1$\\
Voter $i$ has the following ranking:\\
$\text{some candidate}\succ B \succ A \succ \text{some candidate} \succ C $\\
By moving $B$ to the top, $B$ now has $\alpha-1$ many points. If $i$ moves $A$ back one spot, then $A$ now has $\alpha-1$ points too. We define this as having manipulated successfully, since the winner now depends on the tie breaking we perform. But if $i$ were to move $A$ further back still, then $C$ would have to move up, thus obtaining $\alpha$ many points and winning clearly. But $C$ is dispreferred by $i$, so this manipulation is not desirable for her.\\ In order to exclude cases like this one, we therefore need to check at each step what happens with the other candidates that could also become winners. On the other hand, we can move $B$ up with no further considerations, because this will at most hurt other candidates' Borda-score.
\item If we find one candidate in the \textit{possible winners} that a) some voter prefers to actual winner b) will become the winner by moving him to the top and the \textit{actual winner} at some spot behind what he previously occupied, then declare the profile manipulable. Otherwise, declare it non-manipulable.
\end{enumerate}
\item \textit{Copeland}: This works essentially the same as what we use for Borda-manipulability. The only difference is that, instead of the Borda scoring vector, we use the one we induced to determine the winner for the specific profile under consideration.
\end{itemize}
\section{Descriptive Analysis \& Logistic Regression}
Figures $1$ and $2$ show the frequencies of manipulable profiles for $100$ profiles between $10$ and $1000$ voters.
\subsection{Descriptive Analysis}
\begin{SCfigure}[0.5][h]
\caption{Fraction of manipulable profiles on the x-axis plotted against the number of voters on the y-axis, for each voting rule. Each datapoint represents $100$ profiles for $3$ candidates.}
\includegraphics[scale=0.35]{3candidates}
\end{SCfigure}

\begin{SCfigure}[0.5][h]
\caption{Fraction of manipulable profiles on the x-axis plotted against the number of voters on the y-axis, for each voting rule. Each datapoint represents $100$ profiles for $6$ candidates.}
\includegraphics[scale=0.35]{6candidates}
\end{SCfigure}

As the plots already give away, there is no difference discernible between the Borda and the Copeland manipulability. We therefore present them together. 

The following table summarizes the two plots by giving the average fraction of manipulable profiles across all number of voters for the voting rules.

\begin{table}[h]
\centering
\caption{Average Manipulable Fraction of Profiles for $3$ and $6$ Candidates}
\label{my-label}
\begin{tabular}{|l|l|l|}
\hline
               & Average Manipulable Fraction: 3 Candidates & Average Manipulable Fraction: 6 Candidate \\ \hline
Plurality      & $0.1195$                                   & $0.2264$                                  \\ \hline
Borda/Copeland & $0.075$                                     & $0.1634$                                  \\ \hline
\end{tabular}
\end{table}

The data suggests that the more voters a profile has, the less prone to manipulation it becomes. This is indicated by both Figure $1$ and $2$ where the fraction of manipulable profiles clearly decreases for all voting rules as we increase the number of voters. Intuitively this accords well our choice to focus on the scoring-aspect of voting rules in our construction of algorithms in Section 2. The more voters there are, the more points can be handed out. The more points distributed across the candidates, the bigger the variance between the candidates' final score is likely to be. But if these distances are great, it becomes harder for individual voters to manipulate the outcomes. Thus, we see a decrease in manipulability for higher number of voters; or so our attempt at explaining the data goes.

A second observation concerns the effect the number of candidates has on manipulability. The Table $1$ provides some, albeit very limited, evidence that more candidates imply increased manipulability. Whereas for $3$ candidates the average fraction of manipulable profiles under the plurality rule is at roughly $0.12$, for $6$ candidates the number is almost around $10$ percentage points higher. For Borda and Copeland the difference is even more marked. For $6$ candidates the fraction is more than double as big.

Lastly, our data gives good evidence that Copeland and Borda manipulability are highly correlated, if not completely equivalent. At least for the data generated and analyzed a profile was always manipulable under both Borda and Copeland or under none of the two. This either points to a fundamental similarity of the two or to a flaw in the construction of our algorithms.
\subsection{Logistic Regression}
We used the algorithms described earlier to generate a large amount of data on manipulability. Next, we would like to know how probable it is that a profile with a given number of candidates and voters is manipulable under our three voting rules. Furthermore, we would like to compare the impact the choice of voting rule has on manipulability and how high our confidence in this impact should be as compared to pure accident. Finally, we would like to compare the importance of voting rules with respect to man...(line truncated)...

The adequate tool to answer these questions is regression analysis. Since our dependent variable (manipulable? yes/no) is a binary, categorical variable, we should not use simple linear regression since it is unlikely to yield a good fit. Instead the method of choice is logistic regression.

Logistic regression makes use of the standard logistic function to turn binary output into real output. Namely \[F(x)=\frac{1}{1+e^{-(\beta_0+\beta_1+...)}}\] yields a regression model where F(x) can be interpreted as the probability of
\section{Discussion}
\section{Conclusion}
\end{document}