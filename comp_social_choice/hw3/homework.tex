\documentclass[12pt]{article}

\usepackage[margin=1in]{geometry}
\usepackage{amsmath,amsthm,amssymb}
\usepackage{tikz} % for drawing stuff
\usepackage{xcolor} % for \textcolor{}
\usepackage{readarray} % for \getargsC{}
\usepackage{graphicx} % disjoint union
\usepackage[utf8]{inputenc}
\usepackage[T1]{fontenc}
\usepackage{hyperref}

% Math sets
\newcommand{\N}{\mathbb{N}}
\newcommand{\Z}{\mathbb{Z}}
\newcommand{\R}{\mathbb{R}}

% Setup of project
\newenvironment{question}[2][Question]{\begin{trivlist}
\item[\hskip \labelsep {\bfseries #1}\hskip \labelsep {\bfseries #2.}]}{\end{trivlist}}
\newenvironment{answer}[2][Answer]{\begin{trivlist}
\item[\hskip \labelsep {\bfseries #1}\hskip \labelsep {\bfseries #2:}]}{\end{trivlist}}
\begin{document}
% math enumerate
\renewcommand{\theenumi}{\roman{enumi}}

% Short hands
\let\oldsum\sum
\renewcommand{\sum}[3]{\oldsum\limits_{#1}^{#2}#3}
\let\oldprod\prod
\renewcommand{\prod}[3]{\oldprod\limits_{#1}^{#2}#3}

% Disjoint union
\newcommand\Dunion{
  \mathop{\mathchoice
    {\ooalign{$\displaystyle\bigcup$\cr\hss\scalebox{.65}{\raisebox{0.45ex}{\sffamily +}}\hss}}
    {\ooalign{$\textstyle\bigcup$\cr\hss\scalebox{.9}{\raisebox{0.5ex}{\tiny\sffamily +}}\hss}}
    {\ooalign{$\scriptstyle\bigcup$\cr\hss\scalebox{.45}{\raisebox{0.3ex}{\sffamily +}}\hss}}
    {\ooalign{$\scriptscriptstyle\bigcup$\cr\hss\scalebox{.38}{\raisebox{0.3ex}{\sffamily +}}\hss}}
    }
}

\title{Homework 3}
\author{Haukur Páll Jónsson\\
Computational Social Choice}

\maketitle

\begin{question}{3}{}

We count the number of single peaked profiles given $m$ candidates and $n$ voters.
\end{question}

\begin{answer}{a)}{Counting}

Given the defintion of single peakedness in class I claim that given a ordering, $>>$ on $X$ there are $2^{m-1}$ possible preference orders giving $(2^{m-1})^n$ possible profiles assuming $n$ voters. I proof this by induction on $m$.

For $m=1$ then the claim clearly holds since there is just one ordering possible and one profile possible, which is single peaked.

For clarity I also show that the claim holds for $m=3$. Then by the claim there are 4 possible preference orders, $\boldsymbol{R_i}$ which do not break single peakedness. For ordering $a >> b >> c$ then we have the possible preference orders $a \succ b \succ c$, $b \succ a \succ c$, $b \succ c \succ a$ and $c \succ b \succ a$.

Now we show that the claim holds for $m+1$. We go in reverse order. Since a preference order is fully determined when we select a candidate which is on either end of $>>$ we start by selecting that candidate. We have two possibilities. For each of these two possibilities we can add those two possibilites to the end of all possible preference orders with $m$ candidates, which by the induction hypotheses, are $2^{m-1}$. Therefore we have $2^{m-1}\cdot 2=2^m$ possible preference orders for $m+1$.

Therefore, we have $2^{m-1}$ possible preference orders given a ordering $>>$. Then each voter can have one of these preferences. Thus we get: $(2^{m-1})^n$ profiles in total, given a ordering.

Now consider that we have $m!$ possible orderings $>>$. You are tempted to say that then we have $m! \cdot (2^{m-1})^n$ single-peaked profiles but in fact we are counting at least double, since for each profile which satisfies a ordering, that profile also satisfies the reverse ordering. Therefore, we can only say that $m! \cdot (2^{m-1})^n/2$ is an upper bound.
\end{answer}

\begin{question}{4}{Decision problem}

Let $N=\{1,2, ..., n\}$ be the voters, $X=\{x_1,x_2,...,x_m\}$ be the alternatives/candidates. Let $F:\mathcal{L}(X)^n \mapsto \mathcal{L}(X)$ be a S.W.F.. $\boldsymbol{R} \in \mathcal{L}(X)^n$. A rank, $p$, is a position in a preference order.
\end{question}

\begin{answer}{a)}{The problem}

The decision problem, Q, which I formulate is:

Instance: S.W.F. $F$, Profile $\boldsymbol{R}$ a candidate $x$ and rank $p$.

Question: Is candidate $x$ ranked at $p$ for $F(\boldsymbol{R})$?

If we have an oracle which solves Q in 1 time unit we can use Q like this:

We place all candidates $x \in X$ to $X_l$ and go through candidates $x \in X_l$ and ask if this candidate $x$ is in ranked first given $F$ and $\boldsymbol{R}$. If we get a $yes$ then we put $x$ in first place of $\mathcal{L}(X)$ and move to the next rank and remove $x$ from $X_l$. If we get a $no$ then we go to the next candidate in $X_l$.

In a worst case scenario we will need to go through all $m$ candidates in $X_l$ and ask Q for the first placement, $m-1$ for the second placement and so on. This gives:
$$\sum{i=1}{m}{i}=\frac{m \cdot (m+1)}{2}=O(m^2)$$
\end{answer}

\begin{answer}{b)}{Still polynomial}

If Q takes polynomial time unit we consider the previous algorithm again in worst case scenario. Then finding the first place takes $m \cdot poly$ time units and the second placement takes $(m-1) \cdot poly$ time units, etc.. Then we get:

$$\sum{i=1}{m}{i \cdot poly}=poly \cdot \sum{i=1}{m}{i}=poly \cdot \frac{m \cdot (m+1)}{2}=O(m^k) \cdot O(m^2)=O(m^k)$$

Therefore, the algorithm still runs in polynomial time.
\end{answer}

\begin{answer}{c)}{Not polynomial}

Claim: If Q is not decidable in $poly$ time then finding $F(\boldsymbol{R})$ is not solvable in $poly$ time.

To show this, I show the contrapositive: If finding $F(\boldsymbol{R})$ is solvable in $poly$ time then Q is decidable in $poly$ time.

Suppose that we have an algorithm in which we can find $F(\boldsymbol{R})$ in $poly$ time. Given that solution, $\mathcal{L}(X)$, we can answer the decision problem Q stated above by simply checking the solution: Is candidate $x$ ranked at $p$ in $F(\boldsymbol{R})$? If we start at the beginning of $F(\boldsymbol{R})$ then go to rank $p$ in $p$ steps and check if $x$ is indeed there. This procedure takes linear time which is less than $poly$ time. Therefore the finding $F(\boldsymbol{R})$ and going through this procedure takes $poly$ time. Therefore, if we can find $F(\boldsymbol{R})$ in $poly$ time then Q is decidable in $poly$ time. Therefore, If Q is not decidable in $poly$ time then finding $F(\boldsymbol{R})$ is not solvable in $poly$ time.
\end{answer}

\begin{question}{5}{Programming}

I did this excercise with Max Rapp and Silvan Hungerbuhler.
\end{question}

\end{document}