\documentclass[12pt]{article}

\usepackage[margin=1in]{geometry}
\usepackage{amsmath,amsthm,amssymb}
\usepackage[utf8]{inputenc}
\usepackage[T1]{fontenc}

\newcommand{\N}{\mathbb{N}}
\newcommand{\Z}{\mathbb{Z}}

\newenvironment{question}[2][Question]{\begin{trivlist}
\item[\hskip \labelsep {\bfseries #1}\hskip \labelsep {\bfseries #2.}]}{\end{trivlist}}
\newenvironment{answer}[2][Answer]{\begin{trivlist}
\item[\hskip \labelsep {\bfseries #1}\hskip \labelsep {\bfseries #2:}]}{\end{trivlist}}
\begin{document}

\let\oldsum\sum
\renewcommand{\sum}[3]{\oldsum\limits_{#1}^{#2}#3}

\title{Homework 2}
\author{Haukur Páll Jónsson\\
Game Theory}

\maketitle

\begin{question}{1}
PoA
\end{question}
\begin{answer}{a)}
asdf
\end{answer}
\begin{answer}{b)}
We construct a game where the Nash equilibrium is the best social welfare outcome, thus $PoA(G)=1$. Thus, we construct a game with one Nash equilibrium which is also has the highest expected utility for all the players. with the highest utility.

\begin{table}[h]
    \begin{tabular}{|l|l|l|}
    \hline
    ~ & L            & R              \\ \hline
    U & $3\backslash 4$ & $5\backslash 6$   \\ \hline
    D & $4\backslash 5$ & $10\backslash 10$ \\ \hline
    \end{tabular}
\end{table}

Not all action profiles have the same \textit{social welfare}.
This game cleary has one pure Nash equilibrium with $a=(D,R)$

\end{answer}
\begin{answer}{c)}
asdf
\end{answer}
\begin{question}{2}
Show that for weakly dominated strategies, the order of elimination can give different results
\end{question}
\begin{answer}{a)}
Here I provide a counter-example where the order of strategy elimination of weakly dominated strategies matters. I.e. depending on what weakly dominated strategy is eliminated first, iterated elimination will give a different result. Take this game for an example.

\begin{table}[h]
    \begin{tabular}{|l|l|l|}
    \hline
    ~ & L            & R              \\ \hline
    U & $3\backslash 4$ & $5\backslash 6$   \\ \hline
    D & $4\backslash 5$ & $10\backslash 10$ \\ \hline
    \end{tabular}
\end{table}

\end{answer}

\begin{question}{4}
Programming. Finding pure Nash and mixed equilbrium.
\end{question}
\begin{answer}{a)}
I did this programming assigment with Grzegorz.
See attachment: $haukur_grzegorz.tar.gz$
\end{answer}

\end{document}