\documentclass[12pt]{article}

\usepackage[margin=1in]{geometry}
\usepackage{amsmath,amsthm,amssymb}
\usepackage[utf8]{inputenc}
\usepackage[T1]{fontenc}
\usepackage{hyperref}

\newcommand{\N}{\mathbb{N}}
\newcommand{\Z}{\mathbb{Z}}

\newenvironment{question}[2][Question]{\begin{trivlist}
\item[\hskip \labelsep {\bfseries #1}\hskip \labelsep {\bfseries #2.}]}{\end{trivlist}}
\newenvironment{answer}[2][Answer]{\begin{trivlist}
\item[\hskip \labelsep {\bfseries #1}\hskip \labelsep {\bfseries #2:}]}{\end{trivlist}}
\begin{document}

\let\oldsum\sum
\renewcommand{\sum}[3]{\oldsum\limits_{#1}^{#2}#3}

\title{Homework 2}
\author{Haukur Páll Jónsson\\
Game Theory}

\maketitle

\begin{question}{1}
Social welfare, NE for congestion games and PoA
\end{question}
\begin{answer}{a)}
We have a game $G=(N,R,A,d)$ which is a congestion game.
In a potential game, the \textit{pure Nash equilibria} is where the P function is maximal. A congestion game is a potential game with P: \\
$$P(a)=-\sum{r\in R}{}{\sum{k=1}{n_r^a}{d_r(a)}}$$

Thus the Nash equilibria is where this function is maximal for our game. In our game we have two resources with delay $10$ and $\sqrt{x}$, which can not be used at the same time, and $n=200$ players. Thus our function is:

$$P(a)=-(\sum{k=0}{n_{r_1}^a}{10}+\sum{j=0}{n_{r_2}^a}{\sqrt{j}})$$
$$n_{r_1}^a+n_{r_2}^a=200$$

or similary we can write:
$$P(a)=-(\sum{k=0}{n_{r_1}^a}{10}+\sum{j=0}{200-n_{r_1}^a}{\sqrt{j}})$$

This function has a maximal when $n_{r_1}^a=99$ and $n_{r_1}^a=100$ with $P(a)=-1661.4629471031476$. Thus those are the Pure Nash equilibrium.

Now we compute the \textit{expected utilitarian social welfare} of this game. We assume $n_{r_1}^a$ players use the route with a constant cost and the rest use the other route:

$$sw(n_{r_1}^a)=-(n_{r_1}^a\cdot 10+(200 - n_{r_1}^a) \cdot \sqrt{200 - n_{r_1}^a})$$

This function has a maximal when $n_{r_1}^a=155.5$, let us use $n_{r_1}^a=156$ ($\href{https://www.wolframalpha.com/input/?i=f(n)%3D-(n*10%2B(200+-+n)*+sqrt(200+-+n)),+n+from+0+to+200}{The function}$). Thus the highest expected utilitarian social welfare is when $156$ take the route which costs $10$ and the rest, $44$, take the other route. Thus the best social welfare is then: $sw(156)=-1851.87$.

If we now compute the expected utilitarian social welfare for the two pure Nash equilibria we get: \\
$sw(100)=-2000$ and $sw(99)=-2005.03$. Thus, the expected utilitarian social welfare under the Nash equilibrium of $99$ is worse than $100$. Then we compute the \textit{price of anarchy}:

$$PoA(G)=\frac{\min_{a^* \in NE }sw(a^*)}{\max_{a \in A}sw(a)}=\frac{sw(99)}{sw(156)}=\frac{-2005.03}{-1851.87}=1.08$$

Where $NE$ is the set of strategies in Nash equilibrium.
\end{answer}
\begin{answer}{b)}
We construct a game where the worst pure Nash equilibrium is equal to the highest expected utilitarian social welfare, then $PoA(G)=1$. Thus, we construct a game with one Nash equilibrium which is also has the highest expected utility for all the players. with the highest utility.

\begin{table}[h]
    \begin{tabular}{|l|l|l|}
    \hline
    R$\backslash$ C & L            & R              \\ \hline
    U & $3\backslash 4$ & $5\backslash 6$   \\ \hline
    D & $4\backslash 5$ & $10\backslash 10$ \\ \hline
    \end{tabular}
\end{table}

Not all action profiles have the same \textit{social welfare}, since $(U,L)$ and $(D,R)$ differ, for an example.
This game cleary has one pure Nash equilibrium, $a=(D,R)$ thus the worst Nash equilibrium is $a=(D,R)$. The best social welfare outcome is $\max_{a \in A}sw(a)$ and in this game $sw(a)$ is simply the sum: $sw(a)=\sum{k=1}{n}{u_k(a)}$ so it is clearly $a=(D,R)$ as well. Thus the social welfare is equal the worst pure Nash equilibrium, thus $PoA(G)=1$.

\end{answer}
\begin{answer}{c)}
Now we show that for an arbitrary $k \in \N $ we can construct a normal-form game $G$ which has $PoA(G)>k$.

We construct a game $G=(N,A,u)$ s.t. $N=\{1,2, ..., n\}$, $A=A_1xA_2x...xA_n$ and for all players i and j $A_i=A_j$ with $A_i=\{a_1, a_2, ..., a_{k+1}\}$ and $u=(u_1, ..., u_n)$ s.t. for all players i and j $u_i=u_j$ and furthermore, $u_i(a_1, ..., a_n)=k$ iff $a_1=a_k$, $a_2=a_k$, ..., $a_n=a_k$ and 0 otherwise, i.e. the "diagonal" has utility equal to the action number and $0$ otherwise.

This game has clearly $k+1$ Nash equilibria, the worst one being $a^1=(a_1, a_1, ..., a_1)$ with utility $1$. The best social welfare outcome is clearly $a^{k+1}=(a_{k+1}, a_{k+1}, ..., a_{k+1})$ with utility $k+1$. Since utilities as positive the $PoA$ is: \\

$$PoA(G)=\frac{\max_{a \in A}sw(a)}{\min_{a^* \in NE }sw(a^*)}=\frac{k+1}{1}=k+1>k$$

Thus, given any $k \in \N$ there exists a game $G$ s.t. $PoA(G)>k$.
\end{answer}
\begin{question}{2}
Show that for weakly dominated strategies, the order of elimination can give different results
\end{question}
\begin{answer}{a)}
Here I provide a counter-example where the order of strategy elimination of weakly dominated strategies matters. I.e. depending on what weakly dominated strategy is eliminated first, iterated elimination will give a different result.

\begin{table}[h]
    \begin{tabular}{|l|l|l|}
    \hline
    R $\backslash$ C & L            & R              \\ \hline
    U & $3\backslash 2$ & $3\backslash 3$   \\ \hline
    D & $2\backslash 3$ & $3\backslash 3$ \\ \hline
    \end{tabular}
\end{table}

We can clearly see that there are no strictly dominated strategies but there are two weakly dominated strategies since $u_R(U) \geq u_R(D)$ and $u_R(U,L) > u_R(D,L)$ and $u_C(R) \geq u_C(L)$ and $u_C(U,R) > u_C(U,L)$. Thus we can eliminate the two strategies $D$ and $L$. If we first eliminate $D$ then we are left with $u_C(U,R) > u_C(U,L)$ thus we can eliminate $L$ and we are left with $(U,R)$. But if we start by eliminating $L$ we eliminated $(D,L)$ so that it is no longer the case that $u_R(U) \geq u_R(D)$ and $u_R(U,L) > u_R(D,L)$ so we cannot eliminate $D$. Thus we can see that the order of elimination of weakly dominated strategies does matter.

\end{answer}

\begin{question}{4}
Programming. Finding pure Nash and mixed equilbrium.
\end{question}
\begin{answer}{a)}
I did this programming assigment with Grzegorz.
See attachment: $haukur\_grzegorz.tar.gz$
\end{answer}

\end{document}